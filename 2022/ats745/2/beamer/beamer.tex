% !TeX root = ./beamer.tex
%%%%%%%%%%%%%%%%%%%%%%%%%%%%%%%%%%%%%%%%%%%%%%%%%%%%%%%%%%%%%%%%%%%%%%
%
% Beamer template with UiT colour scheme
%
%%%%%%%%%%%%%%%%%%%%%%%%%%%%%%%%%%%%%%%%%%%%%%%%%%%%%%%%%%%%%%%%%%%%%%

\documentclass[xcolor=dvipsnames]{beamer} %
\usetheme[progressbar=frametitle,
	titleformat=smallcaps,
	% sectionpage=none,
	numbering=fraction,
	block=fill,
	background=dark]{metropolis}
% The below gives the font from a standard latex article, and should be used in pair.
% \usefonttheme{professionalfonts} % special math fonts
% \usefonttheme{serif} % default family is serif
\usetikzlibrary{arrows,shapes}
% If you want notes on the side:
% \setbeameroption{show notes on second screen=right} % Both
% \setbeamercolor{note page}{fg=yellow!10}
% \setbeamertemplate{note page}{
% 	\settowidth{\leftmargini}{\usebeamertemplate{itemize item}}
% 	\addtolength{\leftmargini}{-2\labelsep}
% 	\pagecolor{black!90}\vfill\insertnote\vfill
% }

\usepackage{utils/definitions}
\usepackage{utils/divide_toc}
\usepackage{utils/beamerouterthemesplit}
\usepackage{utils/citecmd}
\usepackage{utils/footfix}
\usepackage{utils/colors}
\setbeamerfont{footnote}{size=\tiny}

\title[Chemistry in CESM2]{Chemistry in CESM2}
\author{\textsc{Eirik Rolland Enger}}
\date{\vspace{-3mm}December 7, 2022\hfill\includegraphics[width=4cm]{utils/CSU-Official-wrdmrk-Rev-2.png}}
\logo{\vspace{-2mm}\includegraphics[width=15mm]{utils/CSU-Official-wrdmrk-Rev-2.png}\hspace{1mm}}
% \institute{UiT --- The Arctic University of Norway}
\begin{document}
\maketitle

% For every picture that defines or uses external nodes, you'll have to
% apply the 'remember picture' style. To avoid some typing, we'll apply
% the style to all pictures.
\tikzstyle{every picture}+=[remember picture]
% By default all math in TikZ nodes are set in inline mode. Change this to
% displaystyle so that we don't get small fractions.
\everymath{\displaystyle}

% \begin{frame}%,[allowframebreaks]
% 	\frametitle{Overview}
% 	\setbeamertemplate{section in toc}[sections numbered]
% 	\tableofcontents[hideallsubsections]
% \end{frame}

\section{Motivation}

\subsection{SSW}
\begin{frame}{Stratospheric Sudden Warmings}
	\begin{center}
		\copyrightbox[r]{\includegraphics[width=0.85\linewidth]{./assets/Polar-Vortex.png}}
		{\tiny\href{https://www.climate.gov/news-features/blogs/enso/polar-vortex-going-make-you-put-sweater-be-afraid-be-very-afraid\#:~:text=Thus\%2C\%20the\%20tropospheric\%20polar\%20vortex,that\%20separates\%20the\%20air\%20masses.}{climate.gov}}
		% % -> \%
		% # -> \# OR ####
	\end{center}
	\note{
		\begin{itemize}
			\item SSWs, or stratospheric sudden warmings, are when the stratosphere is
			      warmed by many kelvins. This is preceded by a wind reversal of the
			      stratospheric polar vortex. I.e., the westerlies go eastward.
		\end{itemize}
	}
\end{frame}

\begin{frame}{Stratospheric Sudden Warmings}
	\begin{columns}
		\column{0.4\linewidth}
		\centering
		\copyrightbox[r]{\includegraphics[width=\linewidth]{./assets/Polar-Vortex.png}}
		{\tiny\href{https://www.climate.gov/news-features/blogs/enso/polar-vortex-going-make-you-put-sweater-be-afraid-be-very-afraid\#:~:text=Thus\%2C\%20the\%20tropospheric\%20polar\%20vortex,that\%20separates\%20the\%20air\%20masses.}{climate.gov}}
		\column{0.7\linewidth}
		\begin{itemize}
			\item Defined as a wind reversal (eastward) at $ \SI{10}{\hecto\pascal} $ $ (\sim\SI{25}{\kilo\metre}) $, $ \SI{60}{\degree N} $
			\item Big improvement from including updated parametrizations of turbulent
			      mountain stress (TMS), surface stress due to unresolved topography %marsh2013
			\item A lack of stratospheric internal variability without a high-top atmosphere %marsh2013
		\end{itemize}
	\end{columns}
	\note{
		\begin{itemize}
			\item The change in direction is what is used to define their occurrence
			      and frequency.
			\item With the inclusion of turbulent mountain stress from unresolved
			      topography, the SSW's saw a big improvement.
			\item Even though this is also part of the low-top version (CAM), the
			      frequencies are too small, and WACCM get much closer to observation.
		\end{itemize}
	}
\end{frame}

\begin{frame}{Stratospheric Sudden Warmings}
	\begin{center}
		\includegraphics[width=0.8\linewidth]{./assets/ssw-freq.png}
	\end{center}
	\figurecite{gettleman2019}
	\note{
		\begin{itemize}
			\item Here we see frequency of SSW's between 1975 and 2014, in events per
			      year. Three ensemble members in grayscale, the ensemble mean in black
			      and observations in red.
		\end{itemize}
	}
\end{frame}

\subsection{Ozone}
\begin{frame}{Evolution of the Ozone layer}
	\footnotesize
	\begin{itemize}
		\item WACCM6 is able to reproduce the evolution of the ozone layer (also SH
		      polar ozone hole)
		      % \item Lower stratospheric ozone field is dynamically controlled, and vertical
		      %       velocity will impact the total column ozone
		\item Ozone variability in the tropical stratosphere improves on the inclusion
		      of an internally generated quasi-bilennial oscillation (QBO)  % marsh2013
	\end{itemize}
	\begin{center}
		\includegraphics[width=0.85\linewidth]{./assets/ozone-2.png}
	\end{center}
	\figurecite{gettleman2019}
	\note{
		\begin{itemize}
			\item The figure is showing total column ozone in Dobson units from a WACCM6
			      coupled simulation (blue) togheter with a specified SST (AMIP, green),
			      specified dynamics (purple) and observations (black).
			\item If one look at ozone the biggest improvement is in the polar regions.
			      WACCM6 is able to reproduce the evolution of the ozone layer, and also
			      the SH polar ozone hole.
			\item One important improvement is the inclusion of an internally generated
			      QBO. So again, stratospheric winds (but now in the tropical
			      stratosphere) are needed to more precisely simulate a different
			      physical process, ozone variability.
			\item Not shown are the tropics, where there are larger biases. This is due
			      to tropical upwelling; since the lower stratospheric ozone field is
			      dynamically controlled, vertical velocity impact the total column
			      ozone. And, larger vertical velocities in the lower stratosphere are
			      expected to be associated with reduced ozone due to vertical advection
			      of ozone poor air from the troposphere.
		\end{itemize}
	}
\end{frame}

\subsection{Atmospheric Blocking}
\begin{frame}{Atmospheric Blocking}
	\scriptsize
	Frequency of the meridional gradient of 500-hPa geopotential height below a
	threshold of $ \mathrm{GHGS}>0$, $\mathrm{GHGN}<\SI{-5}{\metre/degree} $
	\begin{align*}
		\mathrm{GHGS} & =\frac{Z(\phi_0)-Z(\phi_{\mathrm{S}})}{\phi_0-\phi_{\mathrm{S}}} \\
		\mathrm{GHGN} & =\frac{Z(\phi_{\mathrm{N}})-Z(\phi_0)}{\phi_\mathrm{N}-\phi_0}
		\label{eq:blocking-frequency}
	\end{align*}
	where \(\phi_{\mathrm{N}}=\SI{78.75}{\degree N}+\Delta\),
	\(\phi_{0}=\SI{60}{\degree N}+\Delta\), \(\phi_{\mathrm{S}}=\SI{41.25}{\degree N}+\Delta\)
	and \(\Delta=\SI{-3.75}{\degree},\SI{0}{\degree},\SI{3.75}{\degree}\) \cite{dandrea1998}.
	\note{
		\begin{itemize}
			\item Blocking frequency is another way tropospheric variability improve.
			\item A definition from D'Andrea et al. (1998) defines it as the frequency
			      of the meridional gradient at 500-hPa geopotential height to be below
			      a threshold of $ \SI{-5}{\metre/\degree} $.
			\item A given longitude is locally defined as blocked on a specific day if
			      these conditions are satisfied for at least one of the three deltas.
		\end{itemize}
	}
\end{frame}

\begin{frame}{Blocking Frequency}
	\begin{figure*}
		\begin{center}
			\includegraphics[width=0.95\textwidth]{./assets/blocking-2.png}
		\end{center}
		\label{fig:blocking-frequency}
	\end{figure*}
	\note{
		\begin{itemize}
			\item All simulations coupled to an active ocean. CESM1 has 35 simulations,
			      CESM2-CAM6 5 simulations and CESM-WACCM6 3 simulations.
			\item CESM2 better than CESM1 in the Greenland blocking bump in March--May
			      ($ \SI{30}{\degree W} $)
			\item CESM2 still has a DJF bias in the Atlantic sector.
			\item CESM2-WACCM6 significantly better than CESM2-CAM6 in the Pacific
			      sector during DJF.
			\item With blocking frequency closer to observations, this indicates that
			      stratospherical dynamical processes are important for high-latitude
			      tropospheric climate variability.

		\end{itemize}
	}
\end{frame}

\subsection{Sea Ice}
\begin{frame}{Sea Ice}
	\footnotesize
	\begin{itemize}
		\item The September NH sea ice extent is better in WACCM6 than in CAM6
		\item Less downward surface SW and LW in WACCM6 due to higher LWP\footnote{liquid
			      water path} which in turn is due to higher aerosol number.
	\end{itemize}
	$ \Rightarrow $ Tropospheric aerosol chemistry impacts Arctic sea ice.
	\begin{center}
		\includegraphics[width=0.95\linewidth]{./assets/ice-extent3.png}
	\end{center}
	\figurecite{gettleman2019}
	\note{
		\begin{itemize}
			\item With sea ice, you have an example of a process on the very bottom of
			      the model that is noticeably affected by improving the resolution of
			      the top of the atmosphere component.
			\item Annual mean sea ice extent is similar in CAM6 and WACCM6, both
			      coupled, but WACCM6 has less melt in summer, resulting in thicker ice.
			\item The September NH sea ice extent is higher and in better accordance
			      with observations in WACCM6 than in CAM6.
			\item The sea ice volume is dropping faster in WACCM6, faster than
			      observations, while in CAM6 the rate is better while the volume is
			      smaller.
			\item Have found that WACCM6 has less downward surface SW and LW because of
			      slihtly higher liquid water path. In winter this is around the ice
			      edge, in summer over the ice. This again is a result of higher aerosol
			      number.
			\item We therefore find that tropospheric aerosol chemistry can make an
			      impact on Arctic sea ice.
		\end{itemize}
	}
\end{frame}

\section{Implementation}

\subsection{Extra computations}
\begin{frame}{Computational Cost}
	\begin{table}
		\caption{Approximate costs of running different atmosphere models (From \href{https://www.cesm.ucar.edu/events/tutorials/2019/files/Lecture5a-mills.pdf}{lecture by Mills})}
		\label{tab:atm-model-cost}
		\scriptsize
		\begin{center}
			\begin{tabular}[c]{cccc}
				Configuration & Resolution                    & Chemistry & Core-hours/simulation years \\
				\hline
				CAM6          & $\SI{1}{\degree} $, $ 32 $ L  & CAM       & $ \num{3700} $              \\
				WACCM6        & $\SI{2}{\degree} $, $ 70 $ L  & MA        & $ \num{5400} $              \\
				WACCM6        & $\SI{1}{\degree} $, $ 70 $ L  & TSMLT     & $ \num{22000} $             \\
				WACCM6-SC     & $\SI{1}{\degree} $, $ 70 $ L  & SC        & $ \num{6000} $              \\
				WACCM6-SD     & $\SI{1}{\degree} $, $ 88 $ L  & TSMLT     & $ \num{23000} $             \\
				WACCM5.4      & $\SI{1}{\degree} $, $ 110 $ L & MA        & $ \num{20000} $             \\
				WACCM5.4-SC   & $\SI{1}{\degree} $, $ 110 $ L & SC        & $ \num{9000} $              \\
			\end{tabular}
		\end{center}
	\end{table}
	\note{
		\begin{itemize}
			\item We have now seen some of the improvements one might expect when using
			      WACCM6 in favour of CAM6. But this comes at a cost. Let us see what
			      that might be, before we look more closely at how and what extra is
			      implemented in WACCM6.
			\item CAM is cheapest by quite a lot.
			\item Even the older MA version, which can only urn at nominal two degrees
			      is close.
			\item SC: specified chemistry. Radiatively active chemical species (e.g.,
			      ozone) are prescribed.
			\item SD: specified dynamics. Winds and temperatures are relaxed to a
			      specific set of data (e.g.,  reanalysis from NASA GEOS).
		\end{itemize}
	}
\end{frame}

% \begin{frame}{Similarities}
% 	\begin{itemize}
% 		\item Top of atmosphere energy budget
% 		\item Cloud radiative effects
% 		\item Tropospheric physical processes are the same in CAM6 and WACCM6
% 	\end{itemize}
% \end{frame}

\subsection{Spatial}
% \begin{frame}{Spatial}
% 	\begin{itemize}
% 		% \item From Marsh 2013, 18 pressure levels between surface and 100hPa are identical between WACCM4 and CCSM4. Higher resolution between 100 hPa and CCSM4's top (8 vs. 17)
% 		\item Top goes from $ \SI{40}{\kilo\metre} $ to $ \SI{140}{\kilo\metre} $
% 		\item Vertical resolution is somewhat similar in the troposphere
% 		\item Horizontal resolution is the same
% 	\end{itemize}
% \end{frame}

\begin{frame}{Spatial}
	\begin{center}
		\copyrightbox[r]{
			\includegraphics[width=0.7\linewidth]{./assets/vert-grid-2.png}
		}{\tiny\href{https://www.cesm.ucar.edu/working_groups/Atmosphere/dycore-res/vertical-phase-1.html}{cesm.ucar.edu}}
	\end{center}
	\note{
		\begin{itemize}
			\item The most obvious change is in vertical resolution.
			\item Brown, Purple and Green are just suggested new coordinates.
			\item Circles are WACCM6 and grey are CAM6, equivalent up to about
			      $\SI{100}{\hecto\pascal}$.
		\end{itemize}
		See more at: \url{https://www.cesm.ucar.edu/events/wg-meetings/2018/presentations/amwgjoint/richter.pdf}
	}
\end{frame}

\subsection{Chemistry}
\begin{frame}{Chemistry Versions}
	\begin{itemize}[<+->]
		\item Neutral chemistry model versions of WACCM6
		      \begin{itemize}
			      \item TSMLT (troposphere, stratosphere, mesosphere, lower thermosphere) \textit{default}
			      \item TS (troposphere and stratosphere mechanism)
			      \item MA (Middle atmosphere)\begin{itemize}
				            \item Similar the older WACCM4, available in nominal $ \SI{2}{\degree} $ only %gettleman2019 table 1
				            \item Reduced set of tropospheric reactions
			            \end{itemize}
			      \item MAD (Middle atmosphere with D region ion chemistry)
			            \begin{itemize}
				            \item Adds $ 15 $ positive and $ 21 $ negative ions % gettleman2019
				            \item Thus, below $ \SI{75}{\kilo\metre} $ electrons are no longer the main
				                  negative charge carrier % gettleman2019
			            \end{itemize}
		      \end{itemize}
		\item Additional thermosphere eXtension (WACCM-X)
	\end{itemize}
	\note{
		\begin{itemize}
			\item There are four chemistry packages to choose from when running WACCM6.
			\item TSMLT is the default in WACCM6, with troposphere, stratosphere,
			      mesosphere and lower thermosphere included.
			\item TS is the troposphere and stratosphere mechanism.
			\item MA is the middle atmosphere version, included as a two degree option.
			      It is very similar to the older default package used in WACCM4, and
			      thus also has a reduced set of tropospheric reactions.
			\item But there is more! Mad! Middle atmosphere with D region ion chemistry
			      included, that is.
			\item The D region is often used when talking about the ionosphere, as it is
			      the lowest region in the ionosphere. It sits at around 60-80 km during
			      the day and disappears during night.
			\item This version adds 15 positive and 21 negative ions, making it so that
			      below $ \SI{75}{\kilo\metre} $, electrons are no longer the main
			      negative charge carrier.
			\item Might also mention the eXtended versions while we're at it.
		\end{itemize}
	}
\end{frame}

\begin{frame}{Chemistry in TSMLT}
	\scriptsize
	MAM4 (\textit{Modal Aerosol Model}), also used in CAM6, but WACCM6 adds chemistry.
	\begin{itemize}
		\item Includes the chemical families \ce{O_x}, \ce{NO_x}, \ce{HO_x}, \ce{ClO_x} and \ce{BrO_x}, as well as \ce{CH4}
		\item Allows growth of sulfate aerosols, so the prognostic stratospheric aerosols can increase in width
		\item Maximum altitude of $ \SI{20}{\kilo\metre} $ for eruptions outputting more than $ \SI{3.5}{\tera\gram} $ \ce{SO2}
	\end{itemize}
	MOZART (\textit{Model for OZone And Related chemical Tracers})
	\begin{itemize}
		\item The chemical mechanism in CESM2, available from WACCM6, but also CAM-chem
		\item See \href{https://agupubs.onlinelibrary.wiley.com/action/downloadSupplement?doi=10.1029\%2F2019MS001882&file=jame21103-sup-0003-2019MS001882+Table_SI-S02.pdf}{table 2}\footnote{
			      \url{https://agupubs.onlinelibrary.wiley.com/action/downloadSupplement?doi=10.1029\%2F2019MS001882&file=jame21103-sup-0003-2019MS001882+Table_SI-S02.pdf}}
		      for a complete list of chemical reactions included in CESM2 when run with the TSMLT
		      (troposphere, stratosphere, mesosphere, lower thermosphere) configuration.
	\end{itemize}
	\note{
		\begin{itemize}
			\item In charge of chemistry we find MAM and MOZART, not to be confused with
			      MOSART (MOdel for Scale Adaptive River Transport).
			\item MAM includes the chemical families \ce{O_x}, \ce{NO_x}, \ce{HO_x},
			      \ce{ClO_x} and \ce{BrO_x}, as well as \ce{CH4}.
			\item Also included are prognostic stratospheric aerosols. MAM4 was updated
			      to allow for growth of sulfate aerosols into the coarse, or larger
			      size, mode. This is important to represent aerosol sources (including
			      volcanic emissions).
			\item The \ce{SO2} emissions from volcanic eruptions are derived from
			      Volcanic Emissions for Earth System Models, which is based on
			      observations. These observations then has to be accounted for, which
			      is done by placing a maximum altitude on eruptions outputting more
			      than $ \SI{3.5}{\tera\gram} $ \ce{SO2}. (The maximum altitude is to
			      account for aerosol self-lofting due to in situ absorption of longwave
			      radiation in estimating the initial altitude of large volcanic
			      \ce{SO2} clouds.)
			\item MOZART takes care of the chemical mechanism, and can also be run in
			      CAM-chem. A complete list of chemical reactions can be seen via link.
		\end{itemize}
	}
\end{frame}

\begin{frame}{Stratospheric Aerosols}
	In CAM, stratospheric aerosols are prescribed based on output from previous WACCM
	simulations
	\begin{figure}
		\begin{center}
			\includegraphics[width=0.95\textwidth]{./assets/frame_00009-2.png}
		\end{center}
		\caption*{\scriptsize Aerosol optical depth from stratospheric volcanic eruption in WACCM}
		\label{fig:so2-distribution}
	\end{figure}
    \note{
        \begin{itemize}
            \item Example of \ce{SO2} evolution after an eruption (aerosol optical depth).
        \end{itemize}
    }
\end{frame}

\begin{frame}{Stratospheric Aerosol Optical Depth}
	\begin{figure}
		\begin{center}
			\includegraphics[width=0.70\textwidth]{./assets/saod-2.png}
		\end{center}
		\caption*{\scriptsize Stratospheric aerosol optical depth at different locations agree well}
		\label{fig:saod}
	\end{figure}
    \vspace{-9mm}
	\figurecite{gettleman2019}
    \note{
        \begin{itemize}
            \item Second example of aerosol optical depth, this time of how well is
                  agrees with observations at different latitudes.
        \end{itemize}
    }
\end{frame}

\begin{frame}{Lumping}
	\begin{itemize}
		\item TSMLT has 231 solution species
		\item Species are lumped togheter to reduce the computational cost
		\item Example: \ce{C10H16} in MOZART-4 turned into five new lumped species, with
		      \ce{APIN}, \ce{BPIN}, \ce{LIMON}, \ce{MYRC} and \ce{BCARY} giving the
		      primary degradation rates.
	\end{itemize}
	\note{
		\begin{itemize}
			\item With that many reactions, 231 solution species, you have to draw the
			      line at some point, but where?
			\item Lumping of chemical species is common.
			\item As an example, take \ce{C10H16} which was one lumped species in
			      MOZART-4. This was turned into four monoterpenes and one
			      sesquiterpene, with corresponding expansion of the oxidation scheme.
			      The primary degradation rates were now based on alpha-pinene
			      (\ce{APIN}), beta-pinene (\ce{BPIN}), limonene (\ce{LIMON}), myrcene
			      (\ce{MYRC}) and beta-caryophyllene (\ce{BCARY}).
		\end{itemize}
	}
\end{frame}

\subsection{Solar and Geomagnetics}
\begin{frame}{Solar and Geomagnetics}
	\begin{itemize}
		\item Photoionization and heating rates uses parametrization of Solomon and Qian (2005), with input from the $ F_{10.7} $ index
		\item Ion-pair production rates are prescribed
		\item Low energy electrons included by the parametrized auroral oval model by
		      Roble and Ridley (1994)
		\item Input to the model is HP, hemispheric power, related to the $ \mathrm{K_p} $
		      index:
		      \begin{equation*}
			      HP=\left\{\begin{aligned}
				       & 16.82\exp(0.32\mathrm{K_p})-4.86, & \mathrm{K_p}\leq 7 \\
				       & 153.13+73.4(\mathrm{K_p}-7.0),    & \mathrm{K_p}>7
			      \end{aligned}\right.
			      \label{eq:hemispheric-power}
		      \end{equation*}
		\item Since WACCM3, E region ionosphere is represented with a chemistry
		      consisting of \ce{O+}, \ce{O2+}, \ce{N+}, \ce{N2+}, \ce{NO+}
	\end{itemize}
	\note{
        \begin{itemize}
            \item Photoionization and heating rates at wavelengths shorter than
                  Lyman-$\alpha$, WACCM6 uses the parametrization of Solomon and Qian.
                  This uses the $ F_{10.7} $ index as input.
            \item Ion-pair production rates by galactic cosmic rays, solar protons, and
                  medium-energy electrons are prescribed
            \item For lower-energy electrons that precipitate in the auroral regions,
                  WACCM6 use the parametrized auroral oval model by Roble and Ridley
                  (1994) (implementation described in Marsh et al. (2007)). This model
                  takes as input the power input to the atmopshere from energetic
                  particle bombardment integrated over either the Northern or Southern
                  Hemisphere, known as the Hemispheric Power in gigawatts.
            \item $ HP $ is assumed to be related to the $ \mathrm{K_P} $ index only,
                  show here, formulation by Zhang and Paxton. If one wish to simulate
                  for example solar storms, higher frequency forcing files can also be
                  used.
        \end{itemize}
	}
\end{frame}

\appendix

\nocite{roble1994}
\nocite{solomon2005}
\nocite{garcia2007}
\nocite{marsh2007}
\nocite{marsh2013}
\nocite{mills2016}
\nocite{mills2017}
\nocite{gettleman2019}
\begin{frame}[allowframebreaks,plain]{References}

	\printbibliography[heading=none]

\end{frame}

\end{document}
