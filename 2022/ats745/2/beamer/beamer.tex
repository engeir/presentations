% !TeX root = ./beamer.tex
%%%%%%%%%%%%%%%%%%%%%%%%%%%%%%%%%%%%%%%%%%%%%%%%%%%%%%%%%%%%%%%%%%%%%%
%
% Beamer template with UiT colour scheme
%
%%%%%%%%%%%%%%%%%%%%%%%%%%%%%%%%%%%%%%%%%%%%%%%%%%%%%%%%%%%%%%%%%%%%%%

\documentclass[xcolor=dvipsnames]{beamer} %
\usetheme[progressbar=frametitle,
	titleformat=smallcaps,
	sectionpage=none,
	numbering=fraction,
	block=fill,
	background=dark]{metropolis}
\usefonttheme{professionalfonts}
\usefonttheme{serif} % default family is serif
\usetikzlibrary{arrows,shapes}
% If you want notes on the side:
% \setbeameroption{show notes on second screen=right} % Both
% \setbeamertemplate{note page}{\pagecolor{yellow!5}\vfill\insertnote\vfill}\usepackage{palatino}

\usepackage{utils/definitions}
\usepackage{utils/divide_toc}
\usepackage{utils/beamerouterthemesplit}
\usepackage{utils/citecmd}
\usepackage{utils/footfix}
\usepackage{utils/colors}

\title[MA in CESM2]{Middle Atmosphere Chemistry in CESM2}
\author{\textsc{Eirik Rolland Enger}}
\date{\vspace{-3mm}December 9, 2022\hfill\includegraphics[width=4cm]{utils/CSU-Official-wrdmrk-Rev-2.png}}
\logo{\vspace{-2mm}\includegraphics[width=15mm]{utils/CSU-Official-wrdmrk-Rev-2.png}\hspace{1mm}}
% \institute{UiT --- The Arctic University of Norway}
\begin{document}
\maketitle

% For every picture that defines or uses external nodes, you'll have to
% apply the 'remember picture' style. To avoid some typing, we'll apply
% the style to all pictures.
\tikzstyle{every picture}+=[remember picture]
% By default all math in TikZ nodes are set in inline mode. Change this to
% displaystyle so that we don't get small fractions.
\everymath{\displaystyle}

\begin{frame}%,[allowframebreaks]
	\frametitle{Overview}

	\setbeamertemplate{section in toc}[sections numbered]
	\tableofcontents[hideallsubsections]

\end{frame}

\section{Standard Chemistry}

\begin{frame}{CAM Default Chemistry}
    A blanket of aerosols if you want to simulate them.
\end{frame}

\section{Stats and numbers}

\subsection{How high}
\begin{frame}
	\frametitle{First frame}
\end{frame}

\subsection{When}
\begin{frame}
	\frametitle{Second frame}
\end{frame}

\subsection{sub2}
\begin{frame}
	\frametitle{Second frame}
\end{frame}

\section{Chemical Reactions}

\subsection{Basic}
\begin{frame}{Basics}
	Nothing.
\end{frame}

\section{Use cases}

\subsection{Aerosol spread}
\begin{frame}{\texorpdfstring{\ce{SO2}}{SO2} to Aerosols}
    Here is an animation!
\end{frame}

\begin{frame}{Final Remarks}
    Finally.
\end{frame}

\begin{frame}{Future Projects}
    List of stuff.
\end{frame}

\begin{frame}{SSW}
    Stratospheric Sudden Warmings
\end{frame}

\appendix

% \section{Appendix}
% \input{underscript/appendix.tex}
% \begin{frame}[allowframebreaks,plain]{References}

% \printbibliography[heading=none]

% \end{frame}

\end{document}
